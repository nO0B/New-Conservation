\documentclass[rutwik_proposal.tex]{subfiles}

\begin{document}

\subsection{Nature and people}

In this most recent phase of conservation history, the emphasis has been on revising our views of society and culture to include biodiversity and ecosystem services in them. In the scientific world, this phase has been marked by increasing collaborations between ecologists and social scientists. These collaborations are directed at trying to understand how cultural structures and institutions can be used in developing sustainable and resilient interactions between humans and their natural environments \cite{Mace14, Ostrom10, Carpenter09}. This is still a fairly utilitarian approach since the ultimate goal is still the sustainable use of our natural capital. However, it takes a much more holistic view of the relationship between humans and nature than previous approaches did.

Among conservationists, there has been a growing recognition of the fact that local communities have very different, and often far more inclusive ways of perceiving nature and their role in it than others do. The social norms and other social and cultural structures present in these communities can lead to highly effective community based stewardship of ecosystems and natural resources. This has given rise to the Rights Based Approach to conservation \cite{Campese09} [and peronal communication with Corine Vriesendorp, Director of the Andes-Amazon program at the Field Museum]. The Rights Based Approach (RBA) stresses the importance of including traditionally marginalized populations in decisions about how to manage the land that they subsist on. A lot of effort is put into educating them about their rights as citizens of their countries and as human beings. Conservation efforts guided by this approach are more bottom up than top down. Even in situations that require a behavioral change to protect a particular species or class of species from over-exploitation, conservation efforts are aimed at informing the locals about the impacts of their actions and about the alternatives available to them \cite{Springer09}.

Although these are significant steps in what I consider to be the right direction, they still have severe limitations. Their biggest limitation is that they seem to be addressing a very narrow sector of the human population. The intended targets are usually individuals and communities who live in close proximity to the affected ecosystems. However, there is evidence to suggest that the biggest threats to biodiversity come from those that are farthest from the ecosystems on which they are vitally dependent. This evidence comes from looking at how consumption of natural resources differs across different societies.

Scholars have argued that the scale of the human enterprise is responsible for most of the threats to biodiversity \cite{Ehrlich94}. The impact of the human enterprise can be thought of as an interaction between three measures: population size, per capita affluence (measured by per capita consumption), and the environmental damage caused by technologies used to produce each unit of consumption \cite{Ehrlich71}. There is ample evidence to show that the average consumption (in terms of energy, food, etc.) in developed countries far oustrips that in less developed countries \cite{UScons08, Lenzen99}. Research also clearly shows that this culture of consumerism is on the rise and has been for the past 30-40 years \cite{Schor99}, and that the consumer culture is starting to become deeply ingrained in us \cite{Schor04}. Taking consumption into account, Western Europe and North America pose much more of a threat to biodiversity than does the rest of the world. Many Asian, African, and South American countries that are rich in biodiversity and are therefore more vulnerable to biodiversity losses seem to simply be responding to pressures offset onto them by their more developed, capitalistic, and consumer driven counterparts. In order to achieve our conservation goals, we need to focus our efforts just as much on individuals and societies that comprise the developed countries of the world as we do on the under-developed countries.

One of the major problems in trying to address such issues however, is that we do not understand why people behave the way they do. Why do people make the decisions they make? Why do people from different societies make different decisions in similar situations? How much of our decision making process is influenced by social norms and how much of it is influenced by personal norms? How situational are different social norms? There are no satisfactory answers to any of these questions. In order to design interventions aimed at influencing people's behavior, we need to have a good understanding of why people behave the way they do. Since social norms, considered by some to be the grammar of society \cite{Bicchieri05}, play a critical role in determining people's behavior, we need to understand how social norms work in order to do truly effective conservation.

\section{Social Norms}\label{sec:norms}


\subsection{Theories of Social Norms}\label{subsec:theories}
\subsection{Norm Interventions in Other Arenas}\label{subsec:interventions}

\chapter{Are Norms Useful in Conservation?}\label{ch:usefulness}

\chapter{Field Studies and Experiments}\label{ch:field}

\chapter{Models}\label{ch:models}

\end{document}