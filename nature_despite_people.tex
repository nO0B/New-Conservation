\documentclass[rutwik_proposal.tex]{subfiles}

\begin{document}

\subsection{Nature despite people}\label{subsec:pplharmnature}

The period extending from the 1970s to the turn of the century witnessed a shift from thinking about nature as completely removed from the human enterprise to thinking about nature as being severely harmed by the human enterprise. Probably the most influential work of this period was Rachel Carson's \textit{Silent Spring} published in 1962. Carson's work awakened the United States and the world at large to the environmental impact that our actions could have (in this case, the use of pesticides, especially DDT) \cite{Carson02}. \textit{Silent Spring} had a number of long lasting impacts. It ensured that 'No one since would be able to sell pollution as the necessary underside of progress so easily or uncritically' \cite{Hynes89}. The ecofeminism movement can be traced back to it as well, as can, at least partially, the formation of the Environmental Defense Fund and the Environmental Protection Agency \cite{Hynes89}.

A number of other scholars researched and wrote about related issues as well. E.O. Wilson identified the 'four mindless horsemen of the environmental apocalypse' (over-exploitation, habitat destruction, introductions of non-native species, and the spread of diseases carried by non-native species) as being responsible for most of the species losses that we were and still are facing \cite{Wilson92}. In \textit{The Condor's Shadow}, David Wilcove drew attention to a number of different ways in which mankind endangers nature and how conservationists the world over attempt to protect nature from human activities \cite{Wilcove00}. Other scholars attempted to quantify the deleterious effects of the four mindless horsemen on species diversity and the environment \cite{Ehrlich94, Wilcove98, Brendan11, Vitousek97, Wilcove13, Poland06, Anagnostakis87, Greenberg14}.

The methods that arose out of this phase of conservation history focused on assuaging the past effects of human influence and on figuring out more biodiversity and environmentally friendly ways of conducting our activities. The land sharing (making agricultural land more biodiverse by having various different crops and micro-habitats) versus land sparing (large areas of homogenous agricultural land inter-mixed with areas of protected preserves) argument was one of the major artifacts of this period \cite{Phalan11, Fischer08, Fischer14}. Other proposed solutions included using biocontrols instead of chemical pesticides and insecticides \cite{Handelsmann96} and regulation of harvesting levels. One of the important outcomes from this phase was the growing advocacy for community based instead of top-down, governmentally or otherwise enforced management \cite{Hutton05}.

While this period in the history of conservation marked a significant move in the direction of a healthier perspective on the relationship between people and nature, it also witnessed the popularization of a completely unidirectional perspective on the relationship between humans and nature. The 'nature despite people' line of thinking fails to acknowledge the fact that we humans depend heavily on nature for our sustenance and well being. Viewing humans and civilization as the enemy and only concentrating on the negative influences that we have historically had on other species paints a depressing picture that often turns people off. Many scholars felt the need for a view that recognized the other side of the equation: how nature affects humans and how we depend on nature. This, along with the fact that a lot of the methods proposed by the 'nature despite people' way of thinking were mired in economic, political, and scientific obstacles, set the stage for the next phase in the history of conservation.

\subsection{Nature for People}\label{subsec:ecoservices}

Gretchen Daily and Paul Ehrlich spearheaded this new phase in the history of conservation and found support amongst pillars in the fields of ecology and conservation such as Daniel Janzen \cite{Daily97, Alexander97, Janzen00, Janzen98, Janzen2000}. They advocated for the protection of natural ecosystems because these ecosystems provide valuable services to humans. These services include \cite{Daily97}:
\begin{itemize}
\item purification of air and water
\item mitigation of floods and droughts
\item detoxification and decomposition of waste
\item generation and renewal of soil and fertility
\item pollination of crops and natural vegetation
\item control of the vast majority of potential agricultural pests
\item dispersal of seeds and translocation of nutrients
\item maintenance of biodiversity, from which humanity has derived key elements of its agricultural, medicinal, and industrial enterprise
\item protection from the sun's harmful ultraviolet rays
\item partial stabilization of climate
\item moderation of temperature extremes and the force of winds and waves
\item support of diverse human cultures
\item providing of aesthetic beauty and intellectual stimulation that lift the human spirit.
\end{itemize}
Very broadly, ecosystem services can be separated into four categories \cite{MEA05}:
\begin{itemize}
\item Provisioning services: This refers to the provision of goods such as food, raw materials, and water.
\item Regulating services: These services result in benefits from the regulation of ecosystem processes. Benefits include carbon sequestration, climate regulation, waste and water treatment, and pest and disease control.
\item Cultural services: These are the nonmaterial, often non-consumptive benefits that people enjoy from nature. Examples of such benefits are recreational experiences, cultural, spiritual, or historical experiences, and science and education.
\item Supporting services: These services are necessary for the production of the other ecosystem services. These include nutrient recycling, primary production, and soil formation.
\end{itemize}

Early estimates placed the value of these services at around \$33 trillion/yr \cite{dArge97}. The Millennium Ecosystem Assessment in 2005 provided much of the impetus required for the adoption of this philosophy in the conservation world. A major shift in goals was the emphasis on the protection of valuable ecosystems instead of on individual species and habitats important for biodiversity. The maintenance and promotion of biodiversity was also seen as a key goal of the ecosystem services approach since it was assumed that the continued provision of services depended on the complexity, redundancy, and stability afforded by biodiversity. Paul Ehrlich likens biodiversity to rivets on an airliner's wing \cite{Ehrlichs81}. Since airliners are built to be much stronger than they need to be, popping a rivet here or a rivet there might not make much of a difference to the structural integrity of the wing. However, beyond a certain point, the wing is compromised and the airliner can no longer fly. Similarly, species extinctions could lead to a situation where we can no longer procure from nature the services that we have come to take for granted.

Recent work however, has shown that there are at least some services where species richness might not be all that important and thus, that ecosystem services alone might not be able to provide satisfactory grounds for conserving biodiversity \cite{Winfree15, Ridder08}. Critics have seized on this apparent shortcoming and added it to the already long list of criticisms that have been leveled against the conservation through ecosystem services school of thought. The past few years have witnessed a heated and at times vitriolic debate between the proponents of 'new conservation' (the ecosystem services argument) and advocates of conservation for the sake of protecting nature for itself.

During his time serving as the chief scientist at The Nature Conservancy (TNC), Peter Kareiva, along with his long time colleague and collaborator Michelle Marvier, pushed strongly for the adoption of the conservation for ecosystem services ethic. At the same time, he also derided other efforts based on older ideologies \cite{Kareiva07, Kareiva12}. Some of the criticisms raised by Kareiva were the same ones that I have brought up in the two previous sections. These include: (i) conservation does not take human welfare into account, (ii) conservation rests on the myth of wilderness as pristine, untouched nature, (iii) conservationists wrongly assume that nature is fragile and that it cannot recover from the damage inflicted by human activities, and (iv) past conservation efforts have failed to protect biodiversity \cite{Kareiva07, Kareiva12}. 

Needless to say, this ended up ruffling quite a few feathers and prompted a number of conservationists to speak out against him and TNC. They accused TNC and Peter Kareiva of selling out to corporate interests \cite{Soule14}, of trying to repackage a centuries old approach that was already a big part of conservation efforts \cite{Greenwald13, Doak14}, and of promoting an overly anthropocentric view that could not possibly solve all of the problems that conservationists had been trying to address because it completely marginalized all other concerns \cite{Doak14}. The trouble stemmed from Kareiva claiming that 'nature for people' was the only ideology that we should be using to guide our conservation efforts; that all other schools of thought were useless and were no longer required. To some, this seemed more like a business strategy aimed at trying to attract new investors by disavowing the older, less economically favorable approaches to conservation [Rob Pringle, personal communication]. In practice however, there is no reason why ecosystem services cannot simply be used as another tool in our growing conservation toolbox instead of replacing all the other tools.

Personally, while I agree with the critcism that 'nature for people' is too anthropocentric and therefore, cannot possibly address all of conservation's concerns, I believe that it does have an important role to play in conservation. Putting conservation in terms of ensuring the continued survival and well being of the human race appeals strongly to rationality and policy. Before ecosystem services considerations became so mainstream, conservation was dominated by emotional, cultural, philosophical, and aesthetic appeals; none of which are easily amenable to legal or political instruments. When put in terms of how much money we stand to lose or gain from the protection of important ecosystems however, we can begin employing innstitutions that are already in place to protect and regulate public goods.

However, the ecosystem services argument does seem to be a very short term approach to conservation. If some technological innovations happen to provide the same services at a lower cost that an ecosystem was providing earlier or if the services are no longer required, there would be no reason to continue preserving the ecosystem under the 'nature for people' ethic. Moreover, the services that an ecosystem provides often become apparent only after the degradation of the ecosystem. Most disturbingly, there are studies that suggest that the valuation or monetization of nature might lead to the weakening of intrinsic motivations for conservation that existed prior to the realization of these utilitarian values \cite{Agrawal15}. At best then, an argument from the 'nature for people' perspective could be used to protect an ecosystem that is at risk of immediate destruction or degradation and where the services provided are obvious and quantifiable, but its applications on a long time scale seem somewhat dubious and untenable.

'Nature for people', like 'nature despite people', adopts a fairly unidirectional perspective on the relationship between nature and people. Both approaches fail to recognize that people are just one of the many components of nature, although 'nature for people' probably comes a bit closer. A view propounding nature as subservient to human needs and desires seems to be just as unhealthy as a view depicting humans as the enemies of nature; and just as incorrect as a view portraying humans and nature as wholly separate entities. We are products of a long evolutionary process and are only here on this planet now because this process, through the interactions between its numerous components both extinct and extant, happened to produce conditions that can support us. Similarly, our interactions with the rest of nature creates conditions that favor or hurt other species. Any conservation methods based on ideologies that ignore this bidirectional relationship is doomed to fail.

In recent years, there has been a growing acknowledgement of this bidirectional relationship, giving rise to the last of the phases in the history of conservation, the 'nature and people' phase.

\end{document}