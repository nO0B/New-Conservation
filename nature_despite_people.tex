\documentclass[rutwik_proposal.tex]{subfiles}

\begin{document}

\subsection{Nature despite people}\label{subsec:pplharmnature}

The period extending from the 1970s to the turn of the century witnessed a shift from thinking about nature as completely removed from the human enterprise to thinking about nature as being severely harmed by the human enterprise. Probably the most influential work of this period was Rachel Carson's \textit{Silent Spring} published in 1962. Carson's work awakened the United States and the world at large to the environmental impact that our actions could have (in this case, the use of pesticides, especially DDT) \cite{Carson02}. It ensured that 'No one since would be able to sell pollution as the necessary underside of progress so easily or uncritically' \cite{Hynes89}. \textit{Silent Spring} had a number of long lasting impacts. The ecofeminism movement can be traced back to it, as can, at least partially, the formation of the Environmental Defense Fund and the Environmental Protection Agency \cite{Hynes89}.

A number of other scholars researched and wrote about related issues as well. E.O. Wilson identified the 'four mindless horsemen of the environmental apocalypse' (over-exploitation, habitat destruction, introductions of non-native species, and the spread of diseases carried by non-native species) as being responsible for most of the species losses that we are now facing \cite{Wilson92}. In \textit{The Condor's Shadow}, David Wilcove drew attention to a number of different ways in which mankind endangers nature and how conservationists the world over attempt to protect nature from human activities \cite{Wilcove00}. Other scholars attempted to quantify the deleterious effects of the four mindless horsemen on species diversity and the environment \cite{Ehrlich94, Wilcove98, Brendan11, Vitousek97, Wilcove13, Poland06, Anagnostakis87, Greenberg14}.

The methods that arose out of this phase of conservation history focused on assuaging the past effects of human influence and on figuring out more biodiversity friendly ways of conducting our activities. The land sharing (making agricultural land more biodiverse by having various different crops and micro-habitats) versus land sparing (large areas of homogenous agricultural land inter-mixed with areas of protected preserves) argument was one of the major artifacts of this period \cite{Phalan11, Fischer08, Fischer14}. Other methods included using biocontrols instead of chemical pesticides and insecticides \cite{Handelsmann96} and regulation of harvesting levels among others. One of the important outcomes from this phase was the growing emphasis on community based management \cite{Hutton05}.

While this period in the history of conservation marked a significant move in the direction of a healthier perspective on the relationship between people and nature, the nature despite people perspective is completely unidirectional. It fails to acknowledge the fact that we humans depend heavily on nature for our sustenance and well being. Viewing humans and civilization as the enemy and only concentrating on the negative influences that we have historically had on other species paints a depressing picture that often turns people off. Many scholars felt the need for a view that recognized the other side of the equation: how nature affects humans and how we depend on nature. This, along with the fact that a lot of the methods proposed by the nature despite people way of thinking were mired in economic, political, and scientific obstacles, set the stage for the next phase in the history of conservation.

\subsection{Nature for People}\label{ecoservices}

Gretchen Daily and Paul Ehrlich spearheaded this new and fairly recent phase in the history of conservation and found support amongst pillars in the fields of ecology and conservation such as Daniel Janzen \cite{Daily97, Alexander97, Janzen00, Janzen98, Janzen2000}. They advocated for the protection of natural ecosystems because these ecosystems provide valuable services to humans. These services include \cite{Daily97}:
\begin{itemize}
\item purification of air and water
\item mitigation of floods and droughts
\item detoxification and decomposition of waste
\item generation and renewal of soil and fertility
\item pollination of crops and natural vegetation
\item control of the vast majority of potential agricultural pests
\item dispersal of seeds and translocation of nutrients
\item maintenance of biodiversity, from which humanity has derived key elements of its agricultural, medicinal, and industrial enterprise
\item protection from the sun's harmful ultraviolet rays
\item partial stabilization of climate
\item moderation of temperature extremes and the force of winds and waves
\item support of diverse human cultures
\item providing of aesthetic beauty and intellectual stimulation that lift the human spirit.
\end{itemize}
Very broadly, ecosystem services can be separated into four categories \cite{MEA05}:
\begin{itemize}
\item Provisioning services: This refers to the provision of goods such as food, raw materials, and water.
\item Regulating services: These services result in benefits from the regulation of ecosystem processes. Benefits include carbon sequestration, climate regulation, waste and water treatment, and pest and disease control.
\item Cultural services: These are the nonmaterial, often non-consumptive benefits that people enjoy from nature. Examples of such benefits are recreational experiences, cultural, spiritual, or historical experiences, and science and education.
\item Supporting services: These services are necessary for the production of the other ecosystem services. These include nutrient recycling, primary production, and soil formation.
\end{itemize}

Early estimates placed the value of these services at around \$33 trillion/yr \cite{dArge97}. The Millennium Ecosystem Assessment provided much of the impetus rquired in the adoption of this philosophy in the conservation world.

\section{Social Norms}\label{sec:norms}
\subsection{Theories of Social Norms}\label{subsec:theories}
\subsection{Norm Interventions in Other Arenas}\label{subsec:interventions}

\chapter{Are Norms Useful in Conservation?}\label{ch:usefulness}

\chapter{Field Studies and Experiments}\label{ch:field}

\chapter{Models}\label{ch:models}

\end{document}