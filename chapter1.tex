\documentclass[rutwik_proposal.tex]{subfiles}

\begin{document}

\chapter{Literature Review}\label{ch:litrev}

Over the past two years, my ideas have evolved rapidly in response to the literature I have come across and because of the conversations I have had with experts in disparate fields. In this chapter, I will present and discuss some of the literature that has influenced me most.

My thoughts have been heavily influenced by reading about different approaches to conservation that have been undertaken and by what I consider to be their drawbacks. These perceived drawbacks might at times have something to do with the underlying philosophies or with the methods used by these different approaches, but their biggest shortcoming is that they have largely been unable to achieve their desired outcomes on a large scale. I believe that this is because these approaches do not take societies and social forces into consideration even though the issues that conservationists have to deal with seem to be a direct result of the behavior of societies. In order to address this, we need to understand how societies work; why they behave how they behave and why individuals in societies make the decisions that they do.

This chapter will provide a brief introduction to some of the major schools of thought in conservation, the methods they have used, and the shortcomings of these methods. It will also introduce social norms, which are probably the most significant social forces that drive the behavior of individuals within societies, and thus, of societies themselves. Through this chapter, I hope to motivate the questions that I hope to begin answering during my time here at Princeton.

\section{A Brief History of Conservation}\label{sec:history}
My name is Rutwik

\section{Social Norms}\label{sec:norms}
\subsection{Theories of Social Norms}\label{subsec:theories}
\subsection{Norm Interventions in Other Arenas}\label{subsec:interventions}

\chapter{Are Norms Useful in Conservation?}\label{ch:usefulness}

\chapter{Field Studies and Experiments}\label{ch:field}

\chapter{Models}\label{ch:models}

\end{document}