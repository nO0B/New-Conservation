\documentclass[rutwik_proposal.tex]{subfiles}

\begin{document}

\chapter{A Model for the Evolution of Societies}\label{ch:models}

The model of social norms and society that I am hoping to explore both mathematically and experimentally borrows its components from a few different areas. The model operates on two scales: intra and inter group. Clearly, this is a model based on the social identity theory (section \ref{subsec:socidentity}). Details of the model are nowhere close to being adequately fleshed out yet, but I will attempt to provide a basic skeleton here.

In this model, society is comprised of a number of different groups. The groups overlap to different extents, i.e., members of society are allowed to belong to multiple groups simultaneously and different groups can share norms, beliefs, and values to differing degrees. Individuals are allowed to migrate between groups. Emigration and immigration between groups is mediated by intra-group dynamics. Also, if a group grows very large, it tends to break up into smaller groups, thus giving rise to new groups in the society \cite{Mullen92}.

At present, the only way of new groups forming is through the fission of large groups. This is fairly unrealistic and restrictive because there are other ways of new groups forming. For example, you could imagine a society in which there are disenchanted individuals in different groups that come together because of their shared disenchantment and thus create a new group. There are plenty of historical precedents for this mechanism. For example, new religions and cults seem to follow this pattern. Usually, the formation of such groups is caused by the influence of a few central powerful figures that are able to rally others to their cause. More reliable information can also give rise to new groups, as can be seen in the anti-cigarette or environmental movements. Currently, I have no good way of fitting such observations into my model.

Also, there will be no dominance relationships between groups to start with; although if I am hoping for any degree of realism, I will somehow have to incorporate group hierarchies into this model. Different norms can be associated with the different groups.

Within a group, individuals and their interactions are conceptualized as nodes and edges on a graph. The topology of the graph is vitally important. Brian Skyrms, in his game theoretic models of the evolution of societies \cite{Skyrms96, Skyrms04}, and Damon Centola, through his online experiments \cite{Centola10}, have both shown how important the initial social structure can be in establishing and maintaining behaviors. I will experiment with different network topologies, but I will also try to characterize what real world interaction networks look like. Details of how I will attempt to do this are provided in the next chapter.

Individuals will make decisions based on rules provided by Ian Couzin and his collaborators' model of decision making in which social information is taken into account by both informed and uninformed individuals \cite{Couzin11}. However, my treatment will be slightly different because the individuals in my society will not all be the same. The information provided by some of the individuals will carry more weight than that provided by others. The reason why I am using this decision making method is because it explicitly takes social information into consideration, which is an important part of the social identity theory. I am assuming a non-homogenous society because I think that this is more realistic and because there is some evidence to suggest that this is actually the case \cite{Koehler16}.

The within group networks are dynamic. Links between nodes are dissolved by both random drift and through experience. If past experiences with an individual have been unpleasant, the individual on the receiving end of the unpleasantness will sever the link. If past experiences have been very pleasant, the probability of the link being dissolved decreases. Pleasantness and unpleasantness are vaguely defined as of now. I am still working on trying to decide on good metrics that do not simply have to do with utility to measure how pleasant a particular interaction is. New links are formed randomly at a low background rate and on the basis of social reputation. Individuals will tend to form links with people that are popular in a society. Social reputation or popularity will simply be a function of connectedness - the more connections an individual has, the more popular he is, and the more likely new members are to connect to him. This should lead to the evolution of scale free networks, and there seems to be some evidence to suggest that a lot of real world networks are scale free \cite{Barabasi09}.

I believe that this model has a few advantages over current models. Firstly, I have not come across any models yet that employ the social identity theory as their starting point. This is surprising given the importance of group membership in determining people's behavior. I am also trying to stay away from arguments of rationality and utility as much as I can in order to see how far I can get without recourse to them. I would like to see if a model that does not assume rationality is as good or better at predicting reality. Finally, and probably most interestingly, a model based on the salience of group memberships will also allow us to examine when and how norms can spread from one group to another. This will have important implications since it could help us focus efforts to change norms in a few groups instead of trying to address all or most of society.

The model is far from complete of course, and has some obvious weaknesses (as pointed out in the discussion above). However, I believe it is important to start modeling norms using a different perspective given that current models have a hard time explaining empirical results. The model presented here would, for instance, easily predict people sanctioning violators of social norms even in one-off interactions if those people belonged to the right groups. On the face of it, it would appear that this model does not allow for any internalization of norms whatsoever, but I think I could get around this objection too by assuming that the nodes in the network have some definite internal states that define proclivity to one or another course of action. These internal states could themselves be subject to very long time scale dynamics.

All in all, even if this model itself does not pan out well, I would still like to explore alternative approaches to modeling the evolution of society and its norms. I am particularly interested in the role that groups and perceived group memberships have to play in this process and would like to investigate other models that look at these so far under-appreciated but immensely important social factors.

\chapter{Field Studies and Experiments}\label{ch:field}

\end{document}