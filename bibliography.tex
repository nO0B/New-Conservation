\documentclass[rutwik_proposal.tex]{subfiles}

\begin{document}

\begin{thebibliography}{100}

\bibitem{Mace14}
Mace, G. M.
(2014).
Whose conservation
\emph{Science 345, 345}(6204),
1558-1560

\bibitem{Pinchot98}
Pinchot, G.
(1998).
\emph{Breaking new ground}.
Island Press.

\bibitem{Pinchot10}
Pinchot, G.
(1910).
\emph{The fight for conservation}.
Doubleday, Page.

\bibitem{Brinkley09}
Brinkley, D.
(2009).
\emph{The wilderness warrior: Theodore Roosevelt and the crusade for America}.
New York: Harper Collins.

\bibitem{Righter05}
Righter, R. W.
(2005)
\emph{The battle over Hetch Hetchy: America's most controversial dam and the birth of modern environmentalism}.
Oxford University Press.

\bibitem{Muir1901}
Muir, J.
(1901).
\emph{Our national parks}.
Houghton Mifflin.

\bibitem{Thoreau06}
Thoreau, H. D., \& Cramer, J.S.
(2006).
\emph{Walden}.
Yale University Press.

\bibitem{Wilson16}
Wilson, E.O.
(2016).
\emph{Half-earth: our planet's fight for life}.
Liveright Publishing Corporation.

\bibitem{Langston95}
Langston, N.
(1995).
\emph{Forest dreams, forest nightmares: The paradox of old growth in the inland west}.
University of Washington Press.

\bibitem{Steinberg13}
Steinberg, T.
(2013).
\emph{Down to Earth: Nature's role in American history}.
Oxford University Press.

\bibitem{Butzer99}
Butzer, K. W.
(1999)
The Indian legacy in the American landscape.
\emph{The American Cities and Technology Reader: Wilderness to Wired City, 3,}
3.

\bibitem{Pyne82}
Pyne, S. J.
(1982)
\emph{Fire in America: A cultural history of wildland and rural fire}.
Princeton University Press.

\bibitem{Agrawal09}
Agrawal, A., \& Redford, K.
(2009).
Conservation and displacement: An overview.
\emph{Conservation and Society, 7(1),} 1.

\bibitem{Dowie06}
Dowie, B. M.
(2006).
The Hidden Cost of Paradise Indigenous people are being displaced to create wilderness areas , to the detriment of all.
\emph{Stanford Social Innovation Review}.

\bibitem{Johnson99}
Johnson, B. H.
(1999).
Conservation , Subsistence , and Class at the Birth of Superior National Forest.\emph{Environmental History, 4(1)},
80–99.

\bibitem{Langston16}
Langston, N.
(2016, January 6).
In Oregon, myth mixes with anger.
\emph{The New York Times}

\bibitem{Stuart04}
Stuart, S. N., Chanson, J. S., Cox, N. A., Young, B. E., Rodrigues, A. S., Fischman, D. L., \& Waller, R. W.
(2004).
Status and trends of amphibian declines and extinctions worldwide.
\emph{Science, 306(5702)},
1783-1786.

\bibitem{Barnosky11}
Barnosky, A. D., Matzke, N., Tomiya, S., Wogan, G. O., Swartz, B., Quental, T. B., ... \& Mersey, B.
(2011).
Has the Earth/'s sixth mass extinction already arrived?
\emph{Nature, 471(7336)},
51-57.

\bibitem{Dirzo14}
Dirzo, R., Young, H. S., Galetti, M., Ceballos, G., Isaac, N. J., \& Collen, B.
(2014).
Defaunation in the Anthropocene.
\emph{Science, 345(6195)},
401-406.

\bibitem{Rodrigues06}
Rodrigues, A.S.L.
(2006).
Are global conservation efforts successful?
\emph{Science 313},
1051–1052.

\bibitem{Hoffman11}
Hoffmann, M. et al.
(2011)
The changing fates of the world’s mammals.
\emph{Philos. Trans. R. Soc. Lond. B: Biol. Sci. 366},
2598–2610.

\bibitem{Hoffman10}
Hoffmann, M. et al.
(2010)
The impact of conservation on the status of the world’s vertebrates.
\emph{Science 330},
1503–1509.

\bibitem{Chape05}
Chape, S. et al.
(2005)
Measuring the extent and effectiveness of protected areas as an indicator for meeting global biodiversity targets.
\emph{Philos. Trans. R. Soc. Lond. B: Biol. Sci. 360},
443–455.

\bibitem{Carson02}
Carson, R.
(2002).
\emph{Silent spring}.
Houghton Mifflin Harcourt.

\bibitem{Hynes89}
Hynes, H. P., \& Carson, R.
(1989).
\emph{The recurring silent spring}.
Pergammon Press.

\bibitem{Wilson92}
Wilson, E. O.
(1992).
\emph{The diversity of life}.
Cambridge(MA): Belknap Press.

\bibitem{Wilcove00}
Wilcove, D. S.
(2000).
\emph{The Condor's shadow: the loss and recovery of wildlife in America}.
Anchor.

\bibitem{Ehrlich94}
Ehrlich, P. R.
(1994). 
Energy Use and Biodiversity Loss. 
\emph{Philosophical Transactions of the Royal Society of London. Series B, Biological Sciences, 344},
99–104.

\bibitem{Wilcove98}
Wilcove, D. S., Rothstein, D., Dubow, J., Phillips, A., \& Losos, E.
(1998). 
Quantifying Threats to Imperiled Species in the United States. 
\emph{BioScience, 48(8)},
607–615.

\bibitem{Brendan11}
Fisher, B., Edwards, D. P., Larsen, T. H., Ansell, F. a., Hsu, W. W., Roberts, C. S., \& Wilcove, D. S. 
(2011). 
Cost-effective conservation: calculating biodiversity and logging trade-offs in Southeast Asia. 
\emph{Conservation Letters, 4(6)}, 
443–450.

\bibitem{Vitousek97}
Vitousek, P. M., D'antonio, C. M., Loope, L. L., Rejmanek, M., \& Westbrooks, R.
(1997). 
Introduced species: a significant component of human-caused global change. 
\emph{New Zealand Journal of Ecology}, 
1-16.

\bibitem{Wilcove13}
Wilcove, D. S., Giam, X., Edwards, D. P., Fisher, B., \& Koh, L. P. 
(2013). 
Navjot's nightmare revisited: logging, agriculture, and biodiversity in Southeast Asia. 
\emph{Trends in ecology \& evolution, 28(9)}, 
531-540.

\bibitem{Poland06}
Poland, T. M., \& McCullough, D. G. 
(2006). 
Emerald ash borer: invasion of the urban forest and the threat to North America’s ash resource. 
\emph{Journal of Forestry, 104(3)}, 
118-124.

\bibitem{Anagnostakis87}
Anagnostakis, S. L. 
(1987). 
Chestnut blight: the classical problem of an introduced pathogen. 
\emph{Mycologia, 79(1)}, 
23-37.

\bibitem{Greenberg14}
Greenberg, J. 
(2014). 
\emph{A feathered river across the sky: the passenger pigeon's flight to extinction}. 
Bloomsbury Publishing USA.

\bibitem{Phalan11}
Phalan, B., Onial, M., Balmford, A., \& Green, R. E. 
(2011). 
Reconciling food production and biodiversity conservation: land sharing and land sparing compared. 
\emph{Science (New York, N.Y.), 333(6047)}, 
1289–91.

\bibitem{Fischer08}
Fischer, J., Brosi, B., Daily, G. C., Ehrlich, P. R., Goldman, R., Goldstein, J., ... \& Ranganathan, J.
(2008). 
Should agricultural policies encourage land sparing or wildlife-friendly farming?. 
\emph{Frontiers in Ecology and the Environment, 6(7)}, 
380-385.

\bibitem{Fischer14}
Fischer, J., Abson, D. J., Butsic, V., Chappell, M. J., Ekroos, J., Hanspach, J., ... \& Wehrden, H. 
(2014). 
Land sparing versus land sharing: moving forward. 
\emph{Conservation Letters, 7(3)}, 
149-157.

\bibitem{Handelsmann96}
Handelsman, J., \& Stabb, E. V. 
(1996). 
Biocontrol of soilborne plant pathogens. 
\emph{The plant cell, 8(10)}, 
1855.

\bibitem{Hutton05}
Hutton, J., Adams, W. M., \& Murombedzi, J. C. 
(2005, December). 
Back to the barriers? Changing narratives in biodiversity conservation. 
\emph{In Forum for development studies}
(Vol. 32, No. 2, pp. 341-370). 
Taylor \& Francis Group.

\bibitem{Daily97}
Daily, G. 
(1997). 
\emph{Nature's services: societal dependence on natural ecosystems}. 
Island Press.

\bibitem{Alexander97}
Alexander, S., Ehrlich, P. R., Goulder, L., Lubchenco, J., Matson, P. A., Mooney, H. A., ... \& Woodwell, G. M. 
(1997). 
\emph{Ecosystem services: benefits supplied to human societies by natural ecosystems} (Vol. 2). 
Washington (DC): Ecological Society of America.

\bibitem{Janzen00}
Janzen, D. H. 
(2000). 
Costa Rica's Area de Conservación Guanacaste: a long march to survival through non-damaging biodevelopment. 
\emph{Biodiversity, 1(2)}, 
7-20.

\bibitem{Janzen98}
Janzen, D. 
(1998). 
Gardenification of Wildland Nature and the Human Footprint*. 
\emph{Science, 279(5355)}, 
1312-1313.

\bibitem{Janzen2000}
Janzen, D. H. 
(2000). 
Costa Rica's Area de Conservación Guanacaste: a long march to survival through non-damaging biodevelopment. 
\emph{Biodiversity, 1(2)}, 
7-20.

\bibitem{dArge97}
d'Arge, R., Limburg, K., Grasso, M., de Groot, R., Faber, S., O'Neill, R. V., ... \& Hannon, B. 
(1997). 
The value of the world's ecosystem services and natural capital.
\emph{Nature, 387},
253-260.

\bibitem{MEA05}
\emph{Ecosystems and human well-being}. 
Vol. 5. 
Washington, DC:: Island press, 2005.

\bibitem{Ehrlichs81}
Ehrlich, P. R., \& Ehrlich, A. H. 
(1981). 
\emph{Extinction: the causes and consequences of the disappearance of species}. 
New York: Random House.

\bibitem{Winfree15}
Kleijn, D., Winfree, R., Bartomeus, I., Carvalheiro, L. G., Henry, M., Isaacs, R., ... \& Ricketts, T. H. 
(2015). 
Delivery of crop pollination services is an insufficient argument for wild pollinator conservation. 
\emph{Nature communications}, 
6.

\bibitem{Ridder08}
Ridder, B. 
(2008). 
Questioning the ecosystem services argument for biodiversity conservation. 
\emph{Biodiversity and Conservation, 17(4)}, 
781-790.

\bibitem{Kareiva07}
Kareiva, P., \& Marvier, M. 
(2007). 
Conservation for the people. 
\emph{Scientific American, 297(4)}, 
50–57.

\bibitem{Kareiva12}
Kareiva, P., \& Marvier, M. 
(2012). 
What Is Conservation Science? 
\emph{BioScience, 62(11)}, 
962–969.

\bibitem{Doak14}
Doak, D. F., Bakker, V. J., Goldstein, B. E., \& Hale, B. 
(2014). 
What is the future of conservation? 
\emph{Trends in Ecology and Evolution, 29(2)}, 
77–81.

\bibitem{Soule14}
Miller, B., Soulé, M. E., \& Terborgh, J. 
(2014). 
‘New conservation’or surrender to development?. 
\emph{Animal Conservation, 17(6)}, 
509-515.

\bibitem{Greenwald13}
Greenwald, N., Dellasala, D. A., \& Terborgh, J. W. 
(2013). 
Nothing New in Kareiva and Marvier. 
\emph{BioScience, 63(4)}, 
241–241.

\bibitem{Agrawal15}
Agrawal, A., Chhatre, A., \& Gerber, E. 
(2015). 
Motivational Crowding in Sustainable Development Interventions. 
\emph{American Political Science Review}, 
734–764.

\bibitem{Ostrom10}
Janssen, M. A., Holahan, R., Lee, A., \& Ostrom, E. 
(2010). 
Lab experiments for the study of social-ecological systems. 
\emph{Science, 328(5978)}, 
613-617.

\bibitem{Carpenter09}
Carpenter, S. R., Mooney, H. A., Agard, J., Capistrano, D., DeFries, R. S., Díaz, S., ... \& Perrings, C. 
(2009). 
Science for managing ecosystem services: Beyond the Millennium Ecosystem Assessment. 
\emph{Proceedings of the National Academy of Sciences, 106(5)}, 
1305-1312.

\bibitem{Campese09}
Campese, J., Sunderland, T., Geriber, T., \& Oviedo, G. (eds.).
(2009).
\emph{Rights-based approaches: Exploring issues and opportunities for conservation}.
CIFOR and IUCN.
Bogor, Indonesia.

\bibitem{Springer09}
Springer, J., \& Studd, K. 
(2009). 
The conversatorio for citizen action. 
\emph{Rights-based approaches}, 
77.

\bibitem{Ehrlich71}
Ehrlich, P. R., \& Holdren, J. P. 
(1971). 
Impact of population growth.

\bibitem{UScons08}
Consumption by the United States.
(2008).
Retrieved from \\http://public.wsu.edu/~mreed/380American\%20Consumption.htm

\bibitem{Lenzen99}
Lenzen, M., \& Smith, S. 
(1999). 
Teaching responsibility for climate change: three neglected issues. 
\emph{Australian Journal of Environmental Education, 15}, 
65-75.

\bibitem{Schor99}
Schor, J. B. 
(1999). 
The Overspent American: Why. 
\emph{New York}.

\bibitem{Schor04}
Schor, J. 
(2004). 
\emph{Born to buy: The commercialized child and the new consumer culture}. 
Simon and Schuster.

\bibitem{Bicchieri05}
Bicchieri, C. 
(2005). 
\emph{The grammar of society: The nature and dynamics of social norms}. 
Cambridge University Press.

\bibitem{Young98}
Young, H. P. 
(1998). 
Social norms and economic welfare. 
\emph{European Economic Review, 42(3)}, 
821-830.

\bibitem{Ellickson91}
Ellickson, R. C.
(1991).
\emph{Order without law: How neighbors settle disputes}.
Harvard University Press

\bibitem{Posner00}
Posner, E. A. 
(2000). 
Law and social norms: The case of tax compliance. 
\emph{Virginia Law Review}, 
1781-1819.

\bibitem{Kinzig13}
Kinzig, A. P., Ehrlich, P. R., Alston, L. J., Arrow, K., Barrett, S., Timothy, G., … Ostrom, E. 
(2013). 
Social Norms and Global Environmental Challenges: The Complex Interaction of Behaviors, Values, and Policy. 
\emph{BioScience, 63(3)}, 
164–175.

\bibitem{Levin11}
Levin, S., \& Ehrlich, P. R. 
(2011). 
The Evolution of Norms. 
\emph{PLoS Biology, 106(6)}, 
1493–1545.

\bibitem{Bicchieri14}
Bicchieri, C., \& Muldoon, R.
(2014).
Social norms.
\emph{The Stanford Encyclopedia of Philosophy}
Spring 2014 Edition.

\bibitem{Parsons51}
Parsons, T. 
(2013). 
\emph{Social system}. 
Routledge.

\bibitem{LaPiere34}
LaPiere, R. T. 
(1934). 
Attitudes vs. actions. 
\emph{Social forces, 13(2)}, 
230-237.

\bibitem{Wicker69}
Wicker, A. W. 
(1969). 
Attitudes versus actions: The relationship of verbal and overt behavioral responses to attitude objects. 
\emph{Journal of social issues, 25(4)}, 
41-78.

\bibitem{Bicchieri09}
Bicchieri, C., \& Xiao, E. 
(2009). 
Do the right thing: but only if others do so. 
\emph{Journal of Behavioral Decision Making, 22(2)}, 
191-208.

\bibitem{Prentice93}
Prentice, D. A., \& Miller, D. T. 
(1993). 
Pluralistic ignorance and alcohol use on campus: some consequences of misperceiving the social norm. 
\emph{Journal of personality and social psychology, 64(2)}, 
243.

\bibitem{Schanck32}
Schanck, R. L. 
(1932). 
A study of a community and its groups and institutions conceived as behaviors of individuals. 
\emph{Psychological Monographs, 43(2)}.

\bibitem{Mackie96}
Mackie, G. 
(1996). 
Ending footbinding and infibulation: A convention account. 
\emph{American sociological review}, 
999-1017.

\bibitem{Shell-Duncan00}
Shell-Duncan, B., \& Hernlund, Y. 
(2000). 
Female circumcision in Africa: Dimensions of the practice and debates. 
\emph{Female “circumcision” in Africa: Culture, controversy, and change}, 
1-40.

\bibitem{Hume1739}
Hume, D.
\emph{A treatise of human nature}.
Oxford University Press.

\bibitem{Tajfel81}
Tajfel, H. 
(1981). 
\emph{Human groups and social categories: Studies in social psychology}. 
Cambridge University Press.

\bibitem{Turner86}
Turner, J. C., \& Oakes, P. J. 
(1986). 
The significance of the social identity concept for social psychology with reference to individualism, interactionism and social influence. 
\emph{British Journal of Social Psychology, 25(3)}, 
237-252.

\bibitem{Zimbardo07}
Zimbardo, P. G. 
(2007). 
\emph{Lucifer Effect}. 
Blackwell Publishing Ltd.

\bibitem{Brewer91}
Brewer, M. B. 
(1991). 
The social self: On being the same and different at the same time. 
\emph{Personality and social psychology bulletin, 17(5)}, 
475-482.

\bibitem{Kramer84}
Kramer, R. M., \& Brewer, M. B. 
(1984). 
Effects of group identity on resource use in a simulated commons dilemma. 
\emph{Journal of Personality and Social Psychology, 46(5)}, 
1044–1057.

\bibitem{Brewer79}
Brewer, M. B. 
(1979). 
In-group bias in the minimal intergroup situation: A cognitive-motivational analysis. 
\emph{Psychological bulletin, 86(2)}, 
307.

\bibitem{Cancian75}
Cancian, F. M. 
(1975). 
\emph{What are norms?: a study of beliefs and action in a Maya community}. 
Cambridge University Press.

\bibitem{Dawes00}
Dawes, R. M., \& Messick, D. M. 
(2000). 
Social dilemmas. 
\emph{International journal of psychology, 35(2)}, 
111-116.

\bibitem{Mullen92}
Mullen, B., Brown, R., \& Smith, C. 
(1992). 
Ingroup bias as a function of salience, relevance, and status: An integration. 
\emph{European Journal of Social Psychology, 22(2)}, 
103-122.

\bibitem{Diamond71}
Diamond, A. S. 
(1971). 
Primitive law, past and present. 
\emph{Routledge}.

\bibitem{Axelrod86}
Axelrod, A.
(1986).
An evolutionary approach to norms.
\emph{The American Political Science Review, 80(4)},
1095-1111.

\bibitem{Nowak06}
Nowak, M. A. 
(2006). 
Five rules for the evolution of cooperation. 
\emph{Science, 314(5805)}, 
1560–3.

\bibitem{Fehr02}
Fehr, E., Fischbacher, U., \& G\"achter, S.
(2002).
Strong reciprocity, human cooperation, and the enforcement of social norms.
\emph{Human Nature, 13(1)},
1-25.

\bibitem{Fehr03}
Fehr, E. \& Fischbacher U.
(2003).
The nature of human altruism.
\emph{Nature, 425},
785-791.

\bibitem{Fehr04}
Fehr, E., \& Fischbacher, U.
(2004).
Third-party punishment and social norms.
\emph{Evolution and Human Behavior, 25},
63-87.

\bibitem{Boyd88}
Boyd, R., \& Richerson, P.
(1988).
The evolution of reciprocity in sizable groups.
\emph{J. theor. Biol., 132},
337-356.

\bibitem{Skyrms96}
Skyrms, B.
(1996).
\emph{Evolution of the social contract}.
Cambridge University Press.

\bibitem{Skyrms04}
Skyrms, B.
(2004).
\emph{The stag hunt and the evolution of social structure}.
Cambridge University Press.

\bibitem{Young1998}
Young, H. P.
(1998).
\emph{Individual strategy and social structure}.
Princeton University Press

\bibitem{Fehr16}
Fehr-Duda, H., \& Fehr, E. 
(2016). 
Game Human Nature.
\emph{Nature, (530)},
413-415.

\bibitem{Allcott11}
Allcott, H. 
(2011). 
Social norms and energy conservation. 
\emph{Journal of Public Economics, 95(9)}, 
1082-1095.

\bibitem{Foer10}
Foer, J. S. 
(2010). 
\emph{Eating animals}. 
Penguin UK.

\bibitem{Silverstone09}
Silverstone, A.
(2009).
\emph{The kind diet}.
Rodale Inc.

\bibitem{Singer06}
Singer, P., \& Mason, J.
(2006).
\emph{The ethics of what we eat: Why our food choices matter}.
Rodale inc.

\bibitem{Centola10}
Centola, D. 
(2010). 
The Spread of Behavior in an Online Social Network Experiment. 
\emph{Science, 329(5996)}, 
1194–1197.

\bibitem{Couzin11}
Couzin, I. D., Ioannou, C. C., Demirel, G., Gross, T., Torney, C. J., Hartnett, a., … Leonard, N. E. 
(2011). 
Uninformed Individuals Promote Democratic Consensus in Animal Groups. 
\emph{Science, 334(6062)}, 
1578–1580.

\bibitem{Koehler16}
Koehler, D. J.
(2016).
Can journalistic ''false balance'' distort public perception of consensus in expert opinion?
\emph{Journal of Experimental Psychology. Applied, 22(1)},
24-38.

\bibitem{Barabasi09}
Barabási, A. L. 
(2009). 
Scale-free networks: a decade and beyond. 
\emph{Science, 325(5939)}, 
412.

\end{thebibliography}

\end{document}