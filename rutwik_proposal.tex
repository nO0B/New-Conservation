\documentclass{report}

\usepackage{subfiles}
\usepackage{graphicx}
\usepackage{epstopdf}
\usepackage{epsfig}
\usepackage{cite}

\title{Social Norms for Conservation}
\author{Rutwik Kharkar}

\begin{document}
\maketitle

\chapter*{Preface}

This report is about my thoughts on why we need to come up with a different and better way of doing conservation. Despite the best efforts of conservationists and preservationists in the past century, we find ourselves in the middle of a biodiversity extinction event of an unprecedented scale and rate. Multiple lines of evidence suggest that this extinction event is caused, in large part, by our consumerist tendencies and our abuse of common pool resources. However, few conservation efforts have tried to deal with these issues that are undeniably more social than legal or political. Thus, I believe that it is imperative that we begin to take the power of social forces into consideration when planning our conservation efforts.

My thoughts have been shaped by conversations I have had with experts in many different fields as well as by the literature that has been produced by these and other experts. Thus, this report will start with a brief literature review of some of the important ideas that I have come across in different fields such as history, conservation biology, and psychology.

The literature review will begin with a brief history of conservation, particularly as it pertains to the United States. In this section, while taking a look at how the field of conservation has evolved, I will point out what I perceive to be the shortcomings of different conservation approaches that have been used in the past and those that are being currently used as well. This will, hopefully, motivate the need for a new way of thinking about how to achieve our desired conservation goals; one that takes into consideration the behaviors of people and the societies that they are a part of.

I will then introduce social norms (section \ref{sec:norms}) as a possible tool that we could use in doing conservation since social norms appear to determine a large part of our social behavior. In this section, I will introduce the different schools of thought on social norms and their predictions for how norms originate, evolve, and change. I will also point out where each of these schools of thought fall short and why we need to do more research on social norms to truly understand them. I will also take a brief look at norm interventions and how they have been used in different situations to change inefficient or harmful social norms.

In the next three chapters, I will introduce my ideas on how I plan to contribute to the knowledge of social norms and how I think they can be used in doing conservation more effectively. 

Chapter \ref{ch:usefulness} will deal with my ideas on determining which social norms we as conservationists need to focus on in order to affect the most significant amount of change. Using the results from this research, I hope to address a concern that many social scientists have voiced, which is, that changing social norms might not make much of a difference to conservation outcomes. Given that different norms often behave very differently and have different properties, these results will also help me in deciding which norms to focus on in researching the questions I will bring up in the final two chapters.

In chapters \ref{ch:field} and \ref{ch:models}, I will discuss how I hope to contribute to the understanding of social norms, particularly conservation related social norms. Chapter \ref{ch:field} will outline some of the field experiments and surveys I hope to undertake in the next few years, and in chapter \ref{ch:models}, I will provide very basic details about some of the models of social norms that I would like to investigate mathematically.

This report is probably best read linearly, but anyone familiar with social norms can skip ahead to section \ref{subsec:interventions} after section \ref{sec:history}. Even for readers familiar with the history of conservation, I would encourage skimming section \ref{sec:history} because my motivation for trying to use the considerable power of social norms derives in large part from the matter presented in that section. While chapters \ref{ch:usefulness}, \ref{ch:field}, and \ref{ch:models} inform each other, they are fairly independent and can be read in any order although I would still recommend reading chapter \ref{ch:usefulness} before the other two.

\chapter*{Acknowledgements}

It is of course always impossible to credit everyone whose guidance and conversations have shaped one's ideas. However, I would at least like to thank those who I know have definitely influenced the way I think about the issues presented in this report. I sincerely apologize to those who have either helped mold my ideas unbeknownst to me or to those who I might forget to mention here.

I believe I would have been quite lost without the support of Professor Debbie Prentice. Despite her busy schedule as Dean of the Faculty at Princeton, she agreed to serve on my committee without hesitation. She also finds time to respond to all my requests for help, regardless of how little sense they might make because of the hurry and excitement in which I send them to her after coming across or coming up with new ideas.

Many thanks are due also to my advisor, Professor Simon Levin, who seems to entertain whatever ideas I bring to him with enthusiasm. He  has been very supportive of me exploring different topics even though I have not followed through on most of them. His encouragement has been crucial for my decision to focus on investigating social norms and their application to different kinds of coordination and prosocial behaviors for my doctoral work.

I fear I have failed to make good use of the knowledge and intellectual fertility of my other two committee members, Professors Corina Tarnita and David Wilcove. However, this is something I intend to rectify since the conversations I have had with them have been very thought provoking and helpful as well.

I cannot thank Professor Henry Horn enough for the long discussions on not only the ideas presented here, but also on various other projects and issues that I have considered investigating at different times. A special thanks also to Professor Rob Pringle who introduced me to some of the literature on Rights Based Approaches to Conservation. The philosophy and methods adopted by conservationists championing this approach have had a profound influence on how I think about conservation.

I also have to mention Matthieu Barbier, a postdoc in the Levin lab, whose contribution in getting me to explore alternative modeling approaches. We have both been displeased, to differing degrees, by the emphasis on game theory and on thinking of people characterized as \textit{Homo economicus} when building models to explain social behavior. Without Matthieu, I probably would not have known as much about other possible approaches as I do now. His anthropological views have also helped me to think about things from a very different perspective. Relatedly, I am also grateful to the Levin lab and the EEB department in general for the informative and stimulating discussions I have had with different members of these groups.

Finally, I cannot pass up this opportunity to thank a great friend and mentor, Jane Masterson. It was only because of my fortuitous acquaintance with her that I seriously considered pursuing an advanced degree in ecology. I had been exploring a number of different fields prior to meeting her without ever really thinking about anything related to ecology as a viable career option. Needless to say, I draw a lot of inspiration from her and her thoughts influence mine immensely.

\tableofcontents

\subfile{chapter1}

\subfile{nature_despite_people.tex}

\subfile{nature_and_ppl.tex}

\subfile{bibliography}

\end{document}