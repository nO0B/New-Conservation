\documentclass[rutwik_proposal.tex]{subfiles}

\begin{document}

\chapter{Can Norms Make a Difference?}\label{ch:usefulness}

While it seems logical to conclude that changing behavior on a large scale in developed Western countries could lead to positive conservation outcomes, there is little evidence to support this line of thinking. Paul Ehrlich has argued that the high rate of energy consumption and consumerism in general is responsible for much of our biodiversity loss \cite{Ehrlich94, Ehrlich71}, but his analysis of energy consumption lumps household level and industry level consumption together.

While behavioral interventions have the power to change household energy consumption \cite{Allcott11}, it is unclear how effective this approach will be at reducing industrial energy consumption. Industrial energy consumption is driven, to a large extent, by production demands which are in turn driven by consumer demands. Changing industrial energy consumption will require not only better technology in order to make industrial processes more efficient, but it will also require changing consumer behavior to reduce consumption in general.

When thinking about social norms to achieve environmental goals, the tendency has been to think about encouraging many small behaviors, such as recycling or car pooling. Each of the individual behaviors by themselves might not make much of a difference, but if most members of society practice these behaviors most of the time, the result could be transformative. This is probably the approach we would have to take if we were trying to curb industrial energy consumption as well.

Such an approach has a major problem: it is very difficult to police each of the smaller behaviors. Rewarding someone for throwing a plastic bottle in a recycling bin instead of a trash can or punishing her for doing the opposite every single time she does so is not feasible. Of course, once social norms are in place or if these behaviors become sufficiently internalized, there will be no need for external rewards or punishments. Establishing these behaviors initially, either through norms or through internalization, is the big issue.

A more profitable approach might be to identify behaviors that have the greatest impacts and to try to address those behaviors first. It is also important to be opportunistic. It is far easier to use the momentum of already existing movements and to add further momentum to them instead of trying to start a new initiative from scratch. To this end, I intend to analyze what the conservation impacts of currently popular movements directed at changing large scale behavior might be.

The movement to reduce meat consumption in the United States seems to be garnering increasing support from many different parties. Importantly, it has accumulated a number of popular media personalities as ardent supporters as well. Natalie Portman, Mike Tyson, Al Gore, James Willstrop (my favorite squash player), and many others have publicly voiced their support for the vegetarianism movement. While the environmental and animal rights issues regarding meat consumption have been well explored \cite{Foer10, Silverstone09, Singer06}, the effects on biodiversity have been largely ignored.

In order to start identifying norms and norm interventions that could have a large impact on protecting species and populations from extirpation, I will start by analyzing the impacts of varying levels of meat consumption. The primary variable of interest will be the amount of land needed to support different levels of consumption. However, one could also look at the location of the land and the effects of byproducts from animal facilities to get a better understanding of the true impact of the meat industry on non-human species. I plan on taking a multi-step approach to this part of my thesis:

\begin{itemize}

\item The first step will be to collect data about the amount of land devoted to animal housing facilities and to look at whether converting these lands to greenspaces would lead to an increase in species diversity, populations, or both. This step by itself will be broken up into two parts. The first part will look at the land freed up on a very coarse scale. I will use country-wide data on average species richness and population sizes to arrive at a very rough estimate of whether this land could support more species or populations. For the second part, I will take a closer look at how the location of the land could lead to more specific predictions about whether or not freeing up this land could have favorable outcomes.

\item The second step will be to figure out how much agricultural land could be converted to wilderness or natural areas as a result of not requiring millions of animals to be fed. This step will also include estimating how much land would be required for crops if everyone were to adopt a vegetarian diet. I will then run similar analyses as in the first step to estimate resulting biodiversity outcomes.

\item Finally, I will attempt to build some realism into the above predictions. It is of course impossible to expect every US citizen to become a vegetarian. It is probably also unhealthy to subsist on an entirely vegetarian diet. So, I will attempt to estimate conservation related outcomes at different levels of reduced meat consumption. What would happen if everyone cut their meat consumption in half? What if we only wanted dairy products from the meat industry?

\end{itemize}

In addition to providing the basis for the rest of my thesis, this part of my thesis will also serve to address an important concern brought up by a few different people - that even if we are successful at changing social norms, this might not help in achieving conservation goals. If it does turn out that changing food, energy or other social norms does not prove very beneficial to non-human species, we will have to look to other approaches to do conservation effectively.

\chapter{Models}\label{ch:models}

\chapter{Field Studies and Experiments}\label{ch:field}

\end{document}