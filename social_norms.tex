\documentclass[rutwik_proposal.tex]{subfiles}

\begin{document}

\section{Social Norms}\label{sec:norms}

Social norms dictate what is and what is not acceptable in a society. Like other social phenomena, they are the unforeseen results of interactions between different members of societies. Recently, social norms have been getting increasing attention from scholars in a wide range of fields, including economics \cite{Young98}, law \cite{Ellickson91, Posner00}, and the environmental movement \cite{Kinzig13}. The aim has been to understand how rules governing behavior might evolve in the absence of explicit government or other institutional regulations. Attempts have been made to model the emergence and evolution of norms, but our understanding of norms remains very incomplete \cite{Levin11}. Given that conservation is not, for the most part, subject to market or utilitarian forces, it is unlikely that we are going to be able to make use of explicit regulations in order to achieve all of our conservation goals. Hence, it becomes crucial for us to develop a better understanding of social norms in order to explore alternative means of conserving what we still have left to us.

In general, people tend to conflate social norms with observable behaviors. Others however, think of them only in terms of beliefs and expectations. All of them struggle to explain the observed variance in norm-induced behaviors and conformity to different norms \cite{Bicchieri14}. In this section, I will discuss the three most prevalent ways of thinking about how norms induce behavior. I will point out some of the deficiencies in each of the theories, while mentioning how each of them makes important predictions about the behavior of social norms and how we might be able to use these predictions in guiding conservation policies. The three canonical theories of conformity are: the Socialized Actor Theory, the Social Identity Theory, and the Rational Choice Model.

\subsection{Socialized Actor Theory}\label{subsec:socialization}

The Socialized Actor Theory, as propounded by Talcott Parsons, posits that people behave in ways that are utility maximizing \cite{Parsons51}. However, utility in this case is not defined in terms of money or other explicit values, but in terms of satisfaction. Satisfaction is maximized by minimizing the guilt or pain associated with making choices that are contrary to one's moral system. Through repeated interactions with members of one's society (parents, friends, colleagues, etc.), societal values become ingrained into the individuals comprising a particular society. People's personal values thus become a mere reflection of social values. If members of society are well socialized, they all share a common value system. Expectations about others' behaviors or beliefs play no role in determining an individual's decisions. There is no room for any discrepancy between social norms and personal attitudes in this theory (attitudes refer to dispositions towards objects, whether material or immaterial, that are based on previous experiences with that object \cite{Agrawal15}). Behavior will always conform to social norms since violation of social norms would also lead to a violation of one's own moral system.

While this theory makes no pretensions to explain how a shared value system evolves in the first place or why it might change, it nevertheless makes important predictions about why norms should change, and why and to what extent people conform to them. Some of these predictions have been empirically tested and the results of these experiments have found the theory wanting.

One of the predictions that the rational actor theory makes is that even in situations where there is an observed discrepancy between attitudes and social norms, behavior should conform to personal attitudes since expectations about others' attitudes do not factor into the decision making process in any way. However, experiments have found that stated attitudes are generally poor predictors of behavior. For example, in a much cited field study by Richard LaPiere conducted in the 1930s, he found that even though hotel managers expressed strong anti-China attitudes, they were quite unlikely to behave in ways that manifested these attitudes \cite{LaPiere34}. When approached, the hotel managers seemed more than willing to accomodate Chinese customers. Allan Wicker provided further evidence of this discrepancy through a number of different experiments \cite{Wicker69}.

Instead, people's perception of social norms and expectations are far better predictors of behavior. Bicchieri and Xiao showed that in a laboratory experiment, participants' observations about others' behavior was a much better predictor of their own behavior than were the stated attitudes of the participants \cite{Bicchieri09}. Deborah Prentice and Dale Miller also showed that college students' consumption of alcohol was strongly affected by how much they thought others expected them to consume even if they themselves were not comfortable with consuming as much \cite{Prentice93}. These and other studies seem to clearly show that our behavior is driven much more by what we expect others to do or by what we think others expect us to do than by our own attitudes.

Secondly, according to the socialized actor theory, the only way to change norms is through extended socialization. This would suggest that norms change slowly. However, there have been situations in which long held social norms changed very quickly. For example, the centuries old practice of foot binding in China was overthrown within just a generation \cite{Mackie96}. Fads and fashion trends also tend to change on a relatively fast time scale. The socialized actor theory fails to account for any of these observations.

In addition, the socialization process is most often incomplete. There are large differences between people's personal attitudes and the social norms of the societies to which they belong. In a year long field study, Richard Schanck showed that there were quantitatively significant differences between attitudes that residents of 'Elm Hollow' stated publicly (a measure of perceived social norms) and their privately held attitudes \cite{Schanck32}. Clearly then, the socialized actor theory does not present a complete picture of social norms. On the other hand, it is definitely true that individuals in most societies do share a common value system. Not only are there norms that are inviolable, but there are also norms that people do not ever think about violating. For example, most people would never think about killing another person for no reason. To this extent, some norms are internalized, inviolable, and do not depend on expectations. These norms seem to coincide more with moral norms, which are generally treated as being separate from social norms.

In terms of its applications to conservation, the socialized actor theory serves to remind us that there are beliefs and norms, whether moral or social, on top of which other, efficient or inefficient norms are built. For example, the superiority of humans over all other animals and living organisms is not a belief that is often questioned. It is also a belief that people in developed countries almost universally subscribe to. Norms related to treatment of animals and public use of natural resources can directly be attributed to this underlying belief. The ideal goal for conservationists would then be to question and change or moderate this belief. However, given the cultural, religious, and historical precedents that have established this belief, and that this normative belief is most likely a product of socialization, changing this belief would probably be a very ambitious and highly infeasible goal, at least in the near term. Since the threats facing species and populations are immediate, we need to set ourselves goals that we can achieve on a relevant time scale.

\end{document}